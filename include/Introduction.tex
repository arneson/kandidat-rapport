% CREATED BY DAVID FRISK, 2015
\chapter{Introduction}
% Joakim
\todo[inline]{Digitala företag skall hantera "data"? Kan det vara hantverkarföretag som skall hantera deras kunddata i excel-ark?}
Data is an important part of every digital company of today. The storage, security and accessibility of this data requires a great deal of attention and a large part of a company's budget. These requirements can be fulfilled by a large company but for startups and small companies it can be a difficult task.
\\[0.5cm]
Our data contains important information about everyday life and the sheer size of data is expanding rapidly. Ever since man has been able to generate data, we've wanted to store it and that is what we have done. At first, the problem was physical storage, since it was expensive to store data. But as with many things, the evolution of technology has made it possible to overcome obstacles like that because we've been able to invent smaller, cheaper and more efficient solutions. The most widely used solution is storing data in database management systems, otherwise known as DBMS, or just "databases". There are a wide range of variations but no clear standards or guidelines defined. 
\\[0.5cm]
There has been research regarding this subject for many years now, decades even. The results and conclusions are usually compiled into extensive and formal reports, which may not be suitable for smaller companies to go through. Especially if the company in question needs to go through multiple to find the information they need. The information is out there, but it is a too big of a hassle to find it. 

\section{Purpose}
% Joakim
This project is focused on creating realistic test cases and analyze these on a few carefully selected database management systems. The result is meant to guide and aid companies with selecting the most suitable solution. 
\\[0.5cm]
The results will be compiled into a user friendly web site which should present the findings in a concise and easy fashion. It should provide the user with easy access to complex information and results.

\section{Problem definitions}
% Victor
\todo[inline]{Could me more clearly defined. What will you be looking for in the DBMS. What do the small companies need? How will you compare them? What parameters is important? Återkopplade scalability, accessibility, security.}
The consumption of data has been growing exponentially for each passing year. Most notable is this in the telecom business where the focus of a subscription has been switched from being about calls and texts, to be about the amount of data usage that the user has at his or her disposal. 
\\[0.5cm]
This puts a pressure on the aspiring startups of today (and also tomorrow), the pressure of handling larger and larger quantities of data and also being able to access this data as efficient as possible, in other words accessibility. Storing data is not the main problem in this case, it's more about finding the best solution that matches the expected data traffic with scalability, accessibility and security in mind.
\\[0.5cm]
Going back to the old saying "time is money", many of the startups doesn't have the resources (or time) to do a detailed research of what database solution that is the most suitable for their operation for the time being. Making the right decision regarding the DBMS early on, can be vital for the future of a startup since a drastically increased user base can be fatal for a poorly chosen database system, thus the importance of scalability.

% source of data consumption http://www.cisco.com/c/en/us/solutions/collateral/service-provider/visual-networking-index-vni/mobile-white-paper-c11-520862.html %

\section{Scope}
% Arneson
The theory of different DBMS and their individual strengths and weaknesses is a very large subject. First and foremost, only five different categories will be considered during this project. In each of these categories one implementation will be chosen which will represent the whole category. 
\\[0.5cm]
The next substantial area that will not be considered is different sets hardware. Technologies can behave differently on different hardware, however, to include this would make this project too large. 

\section{Work distribution}
% Erik


